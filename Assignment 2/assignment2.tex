\documentclass{article}
\setlength{\parskip}{\baselineskip}%
\setlength{\parindent}{0pt}%
\usepackage[utf8]{inputenc}
\usepackage{a4wide}
\usepackage{amsmath}
\usepackage{enumitem}
\usepackage{multicol}
\newcounter{countitems}
\newcounter{nextitemizecount}
\newcommand{\setupcountitems}{%
  \stepcounter{nextitemizecount}%
  \setcounter{countitems}{0}%
  \preto\item{\stepcounter{countitems}}%
}
\makeatletter
\newcommand{\computecountitems}{%
  \edef\@currentlabel{\number\c@countitems}%
  \label{countitems@\number\numexpr\value{nextitemizecount}-1\relax}%
}
\newcommand{\nextitemizecount}{%
  \getrefnumber{countitems@\number\c@nextitemizecount}%
}
\newcommand{\previtemizecount}{%
  \getrefnumber{countitems@\number\numexpr\value{nextitemizecount}-1\relax}%
}
\makeatother    
\newenvironment{AutoMultiColItemize}{%
\ifnumcomp{\nextitemizecount}{>}{3}{\begin{multicols}{2}}{}%
\setupcountitems\begin{itemize}}%
{\end{itemize}%
\unskip\computecountitems\ifnumcomp{\previtemizecount}{>}{3}{\end{multicols}}{}}
\usepackage{graphicx}
\usepackage{listings}
\usepackage{relsize}
\usepackage{amsthm}
\usepackage{dsfont}
\usepackage{wrapfig}
\usepackage{fancyhdr} 
\usepackage{hyperref}
\usepackage[svgnames]{xcolor}
\usepackage[top=70pt,bottom=70pt,left=80pt,right=80pt]{geometry}
\usepackage{subcaption}
\usepackage{verbatim}
\usepackage{wrapfig}
\usepackage{tikz}
\usepackage{float}
\usepackage{csquotes}
\usepackage{sectsty}
\usepackage{apacite}
\usepackage{xcolor}
\newcommand\ddfrac[2]{\frac{\displaystyle #1}{\displaystyle #2}}
\DeclareMathOperator*{\argmin}{arg\,min} 
%\sectionfont{\fontsize{12}{15}\selectfont}
\usepackage{todonotes}
\newcommand{\DL}[1]{\todo[linecolor=blue,backgroundcolor=blue!25,bordercolor=blue,inline]{ \textbf{David}: #1 }}

\lstset{
	breaklines=true,
	columns=fullflexible,
	frame=single,
	numbers=left
}
\pagestyle{fancy}
\fancyhf{}
\cfoot{}
\fancyhead[EL]{\nouppercase\leftmark}
\fancyhead[OR]{\nouppercase\rightmark}
\fancyhead[ER,OL]{\thepage}
\renewcommand{\sectionmark}[1]{\markright{\thesection.\ #1}}
\renewcommand{\subsectionmark}[1]{}

\date{\today}

\begin{document}
\begin{flushright}
Natural Computing \\ 
Assignment 2 \today \\ 
\emph{Stijn de Boer, \textit{s7654321} \\ Ron Hommelsheim \textit{s1234567} \\  David Leeftink, \textit{s4496612} }\\\end{flushright}

% 1
\section{Particle Swarm Optimization I}

\begin{enumerate}[label=\alph*)]
    \item \textit{Compute the fitness of each particle} \\ 
    
    \item \textit{What would be the next position and fitness of each particle after one iterationof the PSO algorithm,  when using $\omega$= 2, $\omega$= 0.5, and $\omega$= 0.1?} \\
    
    \item \textit{Explain what is the effect of the parameter $\omega$} \\
    
    \item \textit{Give an advantage and a disadvantage of a high value of $\omega$} \\
    
\end{enumerate}
%2 
\section{Particle Swarm Optimization II}
\textit{Consider a particle “swarm” consisting of a single member. How would it perform in a trivial task such as the minimization of $f(x) =x^2$ when $\omega$ <1,assuming the particle starts with the velocity pointing away from the optimum?} \\ 


%3
\section{Particle Swarm Optimization III}
\textit{Apply and compare the performance of the two algorithms in terms of quantization error  on  Artificial  data set  1  and  on  the  Iris  data set. Un both algorithms, use the true number of clusters as value of the parameter for setting the number of clusters}
%4
\section{Ant Colony Optimisation I}
\begin{enumerate}[label=\alph*)]
    \item \textit{Is ACO for this problem a Competition-Balanced System (CBS)?} \\ 
    
    \item \textit{If a combination of an ACO algorithm and a problem instance is not a CBS, isthe induced bias always harmful? } \\

\end{enumerate}

%5
\section{Ant Colony Optimisation II}
\textit{What results do you expect for an ant colony algorithm that does not use tabu lists?} \\



\end{document}